\documentclass[]{article}
\usepackage{lmodern}
\usepackage{amssymb,amsmath}
\usepackage{ifxetex,ifluatex}
\usepackage{fixltx2e} % provides \textsubscript
\ifnum 0\ifxetex 1\fi\ifluatex 1\fi=0 % if pdftex
  \usepackage[T1]{fontenc}
  \usepackage[utf8]{inputenc}
\else % if luatex or xelatex
  \ifxetex
    \usepackage{mathspec}
  \else
    \usepackage{fontspec}
  \fi
  \defaultfontfeatures{Ligatures=TeX,Scale=MatchLowercase}
\fi
% use upquote if available, for straight quotes in verbatim environments
\IfFileExists{upquote.sty}{\usepackage{upquote}}{}
% use microtype if available
\IfFileExists{microtype.sty}{%
\usepackage{microtype}
\UseMicrotypeSet[protrusion]{basicmath} % disable protrusion for tt fonts
}{}
\usepackage[margin=1in]{geometry}
\usepackage{hyperref}
\hypersetup{unicode=true,
            pdftitle={Activity Fitness Tracker: Exploratory Analysis},
            pdfauthor={Rebecca Elizabeth Kitching},
            pdfborder={0 0 0},
            breaklinks=true}
\urlstyle{same}  % don't use monospace font for urls
\usepackage{color}
\usepackage{fancyvrb}
\newcommand{\VerbBar}{|}
\newcommand{\VERB}{\Verb[commandchars=\\\{\}]}
\DefineVerbatimEnvironment{Highlighting}{Verbatim}{commandchars=\\\{\}}
% Add ',fontsize=\small' for more characters per line
\usepackage{framed}
\definecolor{shadecolor}{RGB}{248,248,248}
\newenvironment{Shaded}{\begin{snugshade}}{\end{snugshade}}
\newcommand{\KeywordTok}[1]{\textcolor[rgb]{0.13,0.29,0.53}{\textbf{{#1}}}}
\newcommand{\DataTypeTok}[1]{\textcolor[rgb]{0.13,0.29,0.53}{{#1}}}
\newcommand{\DecValTok}[1]{\textcolor[rgb]{0.00,0.00,0.81}{{#1}}}
\newcommand{\BaseNTok}[1]{\textcolor[rgb]{0.00,0.00,0.81}{{#1}}}
\newcommand{\FloatTok}[1]{\textcolor[rgb]{0.00,0.00,0.81}{{#1}}}
\newcommand{\ConstantTok}[1]{\textcolor[rgb]{0.00,0.00,0.00}{{#1}}}
\newcommand{\CharTok}[1]{\textcolor[rgb]{0.31,0.60,0.02}{{#1}}}
\newcommand{\SpecialCharTok}[1]{\textcolor[rgb]{0.00,0.00,0.00}{{#1}}}
\newcommand{\StringTok}[1]{\textcolor[rgb]{0.31,0.60,0.02}{{#1}}}
\newcommand{\VerbatimStringTok}[1]{\textcolor[rgb]{0.31,0.60,0.02}{{#1}}}
\newcommand{\SpecialStringTok}[1]{\textcolor[rgb]{0.31,0.60,0.02}{{#1}}}
\newcommand{\ImportTok}[1]{{#1}}
\newcommand{\CommentTok}[1]{\textcolor[rgb]{0.56,0.35,0.01}{\textit{{#1}}}}
\newcommand{\DocumentationTok}[1]{\textcolor[rgb]{0.56,0.35,0.01}{\textbf{\textit{{#1}}}}}
\newcommand{\AnnotationTok}[1]{\textcolor[rgb]{0.56,0.35,0.01}{\textbf{\textit{{#1}}}}}
\newcommand{\CommentVarTok}[1]{\textcolor[rgb]{0.56,0.35,0.01}{\textbf{\textit{{#1}}}}}
\newcommand{\OtherTok}[1]{\textcolor[rgb]{0.56,0.35,0.01}{{#1}}}
\newcommand{\FunctionTok}[1]{\textcolor[rgb]{0.00,0.00,0.00}{{#1}}}
\newcommand{\VariableTok}[1]{\textcolor[rgb]{0.00,0.00,0.00}{{#1}}}
\newcommand{\ControlFlowTok}[1]{\textcolor[rgb]{0.13,0.29,0.53}{\textbf{{#1}}}}
\newcommand{\OperatorTok}[1]{\textcolor[rgb]{0.81,0.36,0.00}{\textbf{{#1}}}}
\newcommand{\BuiltInTok}[1]{{#1}}
\newcommand{\ExtensionTok}[1]{{#1}}
\newcommand{\PreprocessorTok}[1]{\textcolor[rgb]{0.56,0.35,0.01}{\textit{{#1}}}}
\newcommand{\AttributeTok}[1]{\textcolor[rgb]{0.77,0.63,0.00}{{#1}}}
\newcommand{\RegionMarkerTok}[1]{{#1}}
\newcommand{\InformationTok}[1]{\textcolor[rgb]{0.56,0.35,0.01}{\textbf{\textit{{#1}}}}}
\newcommand{\WarningTok}[1]{\textcolor[rgb]{0.56,0.35,0.01}{\textbf{\textit{{#1}}}}}
\newcommand{\AlertTok}[1]{\textcolor[rgb]{0.94,0.16,0.16}{{#1}}}
\newcommand{\ErrorTok}[1]{\textcolor[rgb]{0.64,0.00,0.00}{\textbf{{#1}}}}
\newcommand{\NormalTok}[1]{{#1}}
\usepackage{graphicx,grffile}
\makeatletter
\def\maxwidth{\ifdim\Gin@nat@width>\linewidth\linewidth\else\Gin@nat@width\fi}
\def\maxheight{\ifdim\Gin@nat@height>\textheight\textheight\else\Gin@nat@height\fi}
\makeatother
% Scale images if necessary, so that they will not overflow the page
% margins by default, and it is still possible to overwrite the defaults
% using explicit options in \includegraphics[width, height, ...]{}
\setkeys{Gin}{width=\maxwidth,height=\maxheight,keepaspectratio}
\IfFileExists{parskip.sty}{%
\usepackage{parskip}
}{% else
\setlength{\parindent}{0pt}
\setlength{\parskip}{6pt plus 2pt minus 1pt}
}
\setlength{\emergencystretch}{3em}  % prevent overfull lines
\providecommand{\tightlist}{%
  \setlength{\itemsep}{0pt}\setlength{\parskip}{0pt}}
\setcounter{secnumdepth}{0}
% Redefines (sub)paragraphs to behave more like sections
\ifx\paragraph\undefined\else
\let\oldparagraph\paragraph
\renewcommand{\paragraph}[1]{\oldparagraph{#1}\mbox{}}
\fi
\ifx\subparagraph\undefined\else
\let\oldsubparagraph\subparagraph
\renewcommand{\subparagraph}[1]{\oldsubparagraph{#1}\mbox{}}
\fi

%%% Use protect on footnotes to avoid problems with footnotes in titles
\let\rmarkdownfootnote\footnote%
\def\footnote{\protect\rmarkdownfootnote}

%%% Change title format to be more compact
\usepackage{titling}

% Create subtitle command for use in maketitle
\newcommand{\subtitle}[1]{
  \posttitle{
    \begin{center}\large#1\end{center}
    }
}

\setlength{\droptitle}{-2em}
  \title{Activity Fitness Tracker: Exploratory Analysis}
  \pretitle{\vspace{\droptitle}\centering\huge}
  \posttitle{\par}
  \author{Rebecca Elizabeth Kitching}
  \preauthor{\centering\large\emph}
  \postauthor{\par}
  \predate{\centering\large\emph}
  \postdate{\par}
  \date{04/04/2018}


\begin{document}
\maketitle

\subsection{Introduction}\label{introduction}

It is now possible to collect a large amount of data about personal
movement using activity monitoring devices such as a
\href{http://www.fitbit.com}{Fitbit},
\href{http://www.nike.com/us/en_us/c/nikeplus-fuelband}{Nike Fuelband},
or \href{https://jawbone.com/up}{Jawbone Up}. These type of devices are
part of the ``quantified self'' movement -- a group of enthusiasts who
take measurements about themselves regularly to improve their health, to
find patterns in their behavior, or because they are tech geeks. But
these data remain under-utilized both because the raw data are hard to
obtain and there is a lack of statistical methods and software for
processing and interpreting the data.

This project makes use of data from a personal activity monitoring
device. This device collects data at 5 minute intervals through out the
day. The data consists of two months of data from an anonymous
individual collected during the months of October and November, 2012 and
include the number of steps taken in 5 minute intervals each day.

\section{Loading and preprocessing the
data}\label{loading-and-preprocessing-the-data}

1.Here we want to first load the data set

\begin{Shaded}
\begin{Highlighting}[]
\KeywordTok{setwd}\NormalTok{(}\StringTok{'/Users/rek514/Documents/Data_science/Git_hub/RepData_PeerAssessment1'}\NormalTok{)}

\CommentTok{# Load in packages needed for later}
\KeywordTok{library}\NormalTok{(dplyr)}
\KeywordTok{library}\NormalTok{(lattice)}

\CommentTok{# Stops numbers being shown in Scientific notation and set decimal places to 3}
\KeywordTok{options}\NormalTok{(}\DataTypeTok{scipen=}\DecValTok{99999}\NormalTok{, }\DataTypeTok{digits=}\DecValTok{3}\NormalTok{)}

\CommentTok{# Load dataset into R}
\NormalTok{dataset <-}\StringTok{ }\KeywordTok{read.csv}\NormalTok{(}\StringTok{'activity.csv'}\NormalTok{)}
\CommentTok{# Convert the date column into a date data class}
\NormalTok{dataset$date <-}\StringTok{ }\KeywordTok{as.Date}\NormalTok{(}\KeywordTok{as.character}\NormalTok{(dataset$date, }\DataTypeTok{format =} \StringTok{'%Y/%m/%d'}\NormalTok{))}
\end{Highlighting}
\end{Shaded}

\section{What is the mean total number of steps taken per
day?}\label{what-is-the-mean-total-number-of-steps-taken-per-day}

\begin{enumerate}
\def\labelenumi{\arabic{enumi}.}
\tightlist
\item
  First we need to find the total number of steps per day
\end{enumerate}

\begin{Shaded}
\begin{Highlighting}[]
\CommentTok{# Group the data by date}
\NormalTok{date_grouped <-}\StringTok{ }\KeywordTok{group_by}\NormalTok{(dataset, date)}
\CommentTok{# Calculate the total sum of steps for each day}
\NormalTok{step_total <-}\StringTok{ }\KeywordTok{summarise}\NormalTok{(date_grouped,}\DataTypeTok{Steps=}\KeywordTok{sum}\NormalTok{(steps, }\DataTypeTok{na.rm =} \OtherTok{TRUE}\NormalTok{))}
\end{Highlighting}
\end{Shaded}

\begin{enumerate}
\def\labelenumi{\arabic{enumi}.}
\setcounter{enumi}{1}
\tightlist
\item
  Next we want to generate a histogram of these step totals.
\end{enumerate}

\begin{Shaded}
\begin{Highlighting}[]
\CommentTok{# Create a histogram with the total step data}
\KeywordTok{hist}\NormalTok{(step_total$Steps,}\DataTypeTok{breaks=}\DecValTok{15}\NormalTok{, }\DataTypeTok{col=}\StringTok{'magenta'}\NormalTok{, }\DataTypeTok{main =} \StringTok{'Total Daily Steps'}\NormalTok{,}
     \DataTypeTok{xlab=}\StringTok{'Total Steps'}\NormalTok{, }\DataTypeTok{ylab =}\StringTok{'Frequency'}\NormalTok{)}
\end{Highlighting}
\end{Shaded}

\includegraphics{FitbitTracker_files/figure-latex/histogram-1.pdf}

\begin{enumerate}
\def\labelenumi{\arabic{enumi}.}
\setcounter{enumi}{2}
\tightlist
\item
  Finally we want to find the mean and median number of daily steps
\end{enumerate}

\begin{Shaded}
\begin{Highlighting}[]
\CommentTok{# Calculate the mean and median for the daily step total}
\NormalTok{stepmean <-}\StringTok{ }\KeywordTok{mean}\NormalTok{(step_total$Steps)}
\NormalTok{stepmedian <-}\StringTok{ }\KeywordTok{median}\NormalTok{(step_total$Steps)}
\end{Highlighting}
\end{Shaded}

The \textbf{mean} daily step count is 9354.23 and the \textbf{median} is
10395.

\section{What is the average daily activity
pattern?}\label{what-is-the-average-daily-activity-pattern}

1.First we want to plot the average number of steps taken across the
whole day.

\begin{Shaded}
\begin{Highlighting}[]
\CommentTok{# Group the data by interval}
\NormalTok{interval_grouped <-}\StringTok{ }\KeywordTok{group_by}\NormalTok{(dataset, interval)}
\CommentTok{# Calculate the total sum of steps for each interval during the day}
\NormalTok{intstep_total <-}\StringTok{ }\KeywordTok{summarise}\NormalTok{(interval_grouped,}\DataTypeTok{Steps=}\KeywordTok{mean}\NormalTok{(steps, }\DataTypeTok{na.rm =} \OtherTok{TRUE}\NormalTok{))}
\CommentTok{# Plot a line graph to show the step count across the day}
\KeywordTok{plot}\NormalTok{(}\DataTypeTok{x=}\NormalTok{intstep_total$interval, }\DataTypeTok{y=}\NormalTok{intstep_total$Steps, }\DataTypeTok{type=}\StringTok{'l'}\NormalTok{,}
     \DataTypeTok{main=}\StringTok{'Average step count across the day'}\NormalTok{, }\DataTypeTok{xlab=}\StringTok{'Interval'}\NormalTok{, }\DataTypeTok{ylab=}\StringTok{'Total Steps'}\NormalTok{, }\DataTypeTok{lwd=}\FloatTok{2.5}\NormalTok{, }\DataTypeTok{col=}\StringTok{'blue'}\NormalTok{)}
\end{Highlighting}
\end{Shaded}

\includegraphics{FitbitTracker_files/figure-latex/time-1.pdf}

\begin{enumerate}
\def\labelenumi{\arabic{enumi}.}
\setcounter{enumi}{1}
\tightlist
\item
  Next we want to find out during which time point were the most steps
  taken?
\end{enumerate}

\begin{Shaded}
\begin{Highlighting}[]
\NormalTok{intervalmax <-}\StringTok{ }\NormalTok{intstep_total$interval[}\KeywordTok{which.max}\NormalTok{(intstep_total$Steps)]}
\end{Highlighting}
\end{Shaded}

In this individual, they made the most steps during the 835th interval.

\section{Imputing missing values}\label{imputing-missing-values}

\begin{enumerate}
\def\labelenumi{\arabic{enumi}.}
\tightlist
\item
  Calculate and report the total number of missing values in the dataset
  (i.e.~the total number of rows with 𝙽𝙰s)
\end{enumerate}

\begin{Shaded}
\begin{Highlighting}[]
\NormalTok{missingtotal <-}\StringTok{ }\KeywordTok{sum}\NormalTok{(}\KeywordTok{is.na}\NormalTok{(dataset$steps))}
\end{Highlighting}
\end{Shaded}

The total number of missing values is 2304.

\begin{enumerate}
\def\labelenumi{\arabic{enumi}.}
\setcounter{enumi}{2}
\tightlist
\item
  We now want to replace these missing values with the mean for that
  corresponding interval. These are saved in a new data set.
\end{enumerate}

\begin{Shaded}
\begin{Highlighting}[]
\CommentTok{# Create a new data set}
\NormalTok{na_dataset <-}\StringTok{ }\NormalTok{dataset}
\CommentTok{# Loop through each of the data set entries}
\NormalTok{for (i in }\DecValTok{1}\NormalTok{:(}\KeywordTok{dim}\NormalTok{(na_dataset)[}\DecValTok{1}\NormalTok{])) \{}
  \CommentTok{# Check if the steps value is NA}
  \NormalTok{if (}\KeywordTok{is.na}\NormalTok{(na_dataset$steps[i])) \{}
    \CommentTok{# Find the correct interval index value}
    \NormalTok{interval_index <-}\StringTok{ }\KeywordTok{match}\NormalTok{(na_dataset$interval[i], intstep_total$interval)}
    \CommentTok{# Replace the NA with the interval mean value}
    \NormalTok{na_dataset$steps[i] <-}\StringTok{ }\NormalTok{intstep_total$Steps[interval_index] \}}
  \NormalTok{else \{\}\}}
\end{Highlighting}
\end{Shaded}

4.Finally we want to calculate the total mean and median number of steps
per day and plot the total daily step count in a histogram.

\begin{Shaded}
\begin{Highlighting}[]
\CommentTok{# Group the data by date}
\NormalTok{date_grouped <-}\StringTok{ }\KeywordTok{group_by}\NormalTok{(na_dataset, date)}
\CommentTok{# Calculate the total sum of steps for each day}
\NormalTok{na_step_total <-}\StringTok{ }\KeywordTok{summarise}\NormalTok{(date_grouped,}\DataTypeTok{Steps=}\KeywordTok{sum}\NormalTok{(steps, }\DataTypeTok{na.rm =} \OtherTok{TRUE}\NormalTok{))}

\CommentTok{# Calculate the mean and median for the daily step total}
\NormalTok{new_stepmean <-}\StringTok{ }\KeywordTok{mean}\NormalTok{(na_step_total$Steps)}
\NormalTok{new_stepmedian <-}\StringTok{ }\KeywordTok{median}\NormalTok{(na_step_total$Steps)}

\CommentTok{# Create a histogram with the total step data}
\KeywordTok{hist}\NormalTok{(na_step_total$Steps,}\DataTypeTok{breaks=}\DecValTok{15}\NormalTok{, }\DataTypeTok{col=}\StringTok{'yellow'}\NormalTok{, }\DataTypeTok{main =} \StringTok{'Total Daily Steps'}\NormalTok{,}
     \DataTypeTok{xlab=}\StringTok{'Total Steps'}\NormalTok{, }\DataTypeTok{ylab =}\StringTok{'Frequency'}\NormalTok{)}
\end{Highlighting}
\end{Shaded}

\includegraphics{FitbitTracker_files/figure-latex/recalculate-1.pdf}

The new \textbf{mean} daily step count is 10766.189 and the
\textbf{median} is 10766.189.\\
This clearly shows that the presence of the missing values in the
original data set has shifted the overall mean daily step count down by
1411.959 steps.\\
Filling these missing values in has also meant the current data show a
\textbf{normal distribution} (no skew) where the median and mean are the
same. Previously, the missing values caused the data to show a
\textbf{negative skew}.

\section{Are there differences in activity patterns between weekdays and
weekends?}\label{are-there-differences-in-activity-patterns-between-weekdays-and-weekends}

1.Create a new factor variable in the dataset with two levels --
``weekday'' and ``weekend'' indicating whether a given date is a weekday
or weekend day.

\begin{Shaded}
\begin{Highlighting}[]
\CommentTok{# Add a new empty column and name it 'Weekday'}
\NormalTok{na_dataset[,}\DecValTok{4}\NormalTok{] <-}\StringTok{ }\OtherTok{NA}
\KeywordTok{colnames}\NormalTok{(na_dataset)[}\DecValTok{4}\NormalTok{] <-}\StringTok{ 'Weekday'}

\CommentTok{# Loop through each of the data set entries}

\NormalTok{for (i in }\DecValTok{1}\NormalTok{:(}\KeywordTok{dim}\NormalTok{(na_dataset)[}\DecValTok{1}\NormalTok{])) \{}
  \CommentTok{# Check if the day is a saturday or sunday}
  \NormalTok{if ((}\KeywordTok{weekdays}\NormalTok{(na_dataset$date[i]) ==}\StringTok{ 'Saturday'}\NormalTok{)|(}\KeywordTok{weekdays}\NormalTok{(na_dataset$date[i]) ==}\StringTok{ 'Sunday'}\NormalTok{))\{}
    \CommentTok{# If so, label as a weekend}
    \NormalTok{na_dataset$Weekday[i] <-}\StringTok{  'Weekend'}\NormalTok{\}}
  \CommentTok{# If not a weekend, label as a weekday}
  \NormalTok{else \{na_dataset$Weekday[i] =}\StringTok{ 'Weekday'}\NormalTok{\}\}}

\CommentTok{#Convert to a factor variable}
\NormalTok{na_dataset$Weekday <-}\StringTok{ }\KeywordTok{as.factor}\NormalTok{(na_dataset$Weekday)}
\KeywordTok{head}\NormalTok{(na_dataset)}
\end{Highlighting}
\end{Shaded}

\begin{verbatim}
##    steps       date interval Weekday
## 1 1.7170 2012-10-01        0 Weekday
## 2 0.3396 2012-10-01        5 Weekday
## 3 0.1321 2012-10-01       10 Weekday
## 4 0.1509 2012-10-01       15 Weekday
## 5 0.0755 2012-10-01       20 Weekday
## 6 2.0943 2012-10-01       25 Weekday
\end{verbatim}

Above are the top 6 cases of the new data set with the additional
\textbf{Weekday} variable.

\begin{enumerate}
\def\labelenumi{\arabic{enumi}.}
\setcounter{enumi}{1}
\tightlist
\item
  We now want to see graphically whetehr the average steps at each
  interval period vary between weekend days and weekdays.
\end{enumerate}

\begin{Shaded}
\begin{Highlighting}[]
\CommentTok{# Group data by weekend variable and interval}
\NormalTok{weekday_grouped <-}\StringTok{ }\KeywordTok{group_by}\NormalTok{(na_dataset, Weekday, interval)}
\CommentTok{# Daily step total for each weekday variable and interval}
\NormalTok{weekend_step_total <-}\StringTok{ }\KeywordTok{summarise}\NormalTok{(weekday_grouped,}\DataTypeTok{Steps=}\KeywordTok{mean}\NormalTok{(steps, }\DataTypeTok{na.rm =} \OtherTok{TRUE}\NormalTok{))}

\CommentTok{#Create a panel plot to show weekend and weekday data seperatly.}
\KeywordTok{xyplot}\NormalTok{(weekend_step_total$Steps ~}\StringTok{ }\NormalTok{weekend_step_total$interval |}\StringTok{ }\NormalTok{weekend_step_total$Weekday, }\DataTypeTok{type=}\StringTok{'l'}\NormalTok{, }\DataTypeTok{layout=}\KeywordTok{c}\NormalTok{(}\DecValTok{1}\NormalTok{,}\DecValTok{2}\NormalTok{), }\DataTypeTok{xlab =} \StringTok{'Time Interval'}\NormalTok{, }\DataTypeTok{ylab =} \StringTok{'Total Step Count'}\NormalTok{, }\DataTypeTok{lwd=}\FloatTok{2.5}\NormalTok{, }\DataTypeTok{col=}\StringTok{'red'}\NormalTok{, }\DataTypeTok{main=}\StringTok{'Daily Step Count for the Weekend and Weekdays'}\NormalTok{)}
\end{Highlighting}
\end{Shaded}

\includegraphics{FitbitTracker_files/figure-latex/panel plot-1.pdf}


\end{document}
